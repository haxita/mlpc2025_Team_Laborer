\documentclass{article}

\usepackage[final]{neurips_2023}

\usepackage[utf8]{inputenc}
\usepackage[T1]{fontenc}
\usepackage{hyperref}
\usepackage{url}
\usepackage{booktabs}
\usepackage{amsfonts}
\usepackage{nicefrac}
\usepackage{microtype}
\usepackage{xcolor}
\usepackage{graphicx}
\usepackage{caption}
\usepackage{subcaption}
\usepackage{amsmath}

\title{task 3 and 4}
\author{Dmitrii Troitskii}
\date{May 2025}

\begin{document}

\maketitle

\section*{Task 3: Audio Features}

\subsection*{(a) Selected Subset of Audio Features and Selection Process}

\textbf{Selected Features:}
\begin{itemize}
    \item \textbf{MFCC}: Mel-Frequency Cepstral Coefficients to capture timbral properties.
    \item \textbf{Embeddings}: Learned representations, possibly from a pretrained model.
    \item \textbf{Spectral Contrast}: Captures differences between spectral peaks and valleys.
    \item \textbf{Spectral Flatness}: Measures how noise-like a sound is.
    \item \textbf{Spectral Bandwidth}: Quantifies frequency spread.
    \item \textbf{Mel-Spectrogram}: A perceptually scaled time-frequency representation.
\end{itemize}

\textbf{Selection Process:}
\begin{itemize}
    \item \textbf{Feature Diversity}: Combined complementary features covering timbral, spectral, and perceptual aspects of audio.
    \item \textbf{Empirical Evaluation}: Features were loaded from precomputed `.npz` files and concatenated before dimensionality reduction.
    \item \textbf{Dimensionality Handling}: PCA was used to reduce feature size while preserving 95\% of the variance, which also helps prevent overfitting.
\end{itemize}

\subsection*{(b) Preprocessing Techniques Applied to Audio Features}

\textbf{Preprocessing Steps:}
\begin{itemize}
    \item \textbf{Standardization}: All features were normalized using \texttt{StandardScaler} to have zero mean and unit variance.
    \item \textbf{Dimensionality Reduction}: Principal Component Analysis (PCA) was applied to compress features while retaining 95\% of total variance.
    \item \textbf{Batch Transformation}: The same pipeline was fitted on the training set and then applied to validation and test sets to ensure consistency.
\end{itemize}


\section*{Task 4: Evaluation}

\subsection*{(a) Evaluation Criterion for Comparing Hyperparameter Settings and Algorithms}

\textbf{Chosen Criterion:}
\begin{itemize}
    \item \textbf{F1-Score (Macro-Averaged)}: This metric was used to evaluate and compare model performance. It calculates the F1-score for each class independently and then averages them, ensuring that rare and frequent classes are treated equally.
\end{itemize}

\textbf{Rationale:}
\begin{itemize}
    \item \textbf{Class Imbalance}: Many of the sound events occur infrequently, so metrics like accuracy would not reflect true performance.
    \item \textbf{Balanced Trade-off}: Macro-F1 balances both false positives and false negatives, which is important for multi-label tasks.
    \item \textbf{Cross-Validation Integration}: The F1-score was directly used as the scoring function during hyperparameter tuning via \texttt{GridSearchCV}.
\end{itemize}

\subsection*{(b) Baseline and Best Possible Performance}

\textbf{Baseline Performance:}
\begin{itemize}
    \item \textbf{Definition}: The baseline is defined by a naive model that always predicts the absence of events (all-zero labels).
    \item \textbf{Computation}: Based on the mean positive rate of the validation set across all classes, the baseline accuracy is approximately \( 1 - \text{mean positive rate} \).
    \item \textbf{Purpose}: This baseline reflects how well a model could perform by simply exploiting class imbalance — without learning meaningful patterns.
\end{itemize}

\textbf{Best Possible Performance:}
\begin{itemize}
    \item \textbf{Definition}: The best performance corresponds to the highest macro F1-score obtained through model selection and tuning.
    \item \textbf{Estimation}: This was achieved via \texttt{GridSearchCV} and cross-validation, using various models and hyperparameters. While perfect classification is unlikely due to label noise and overlapping events, the best model significantly outperforms the baseline.
\end{itemize}



\end{document}
